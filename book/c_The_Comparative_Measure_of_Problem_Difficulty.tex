% The Clever Algorithms Project: http://www.CleverAlgorithms.com
% (c) Copyright 2010 Jason Brownlee. Some Rights Reserved. 
% This work is licensed under a Creative Commons Attribution-Noncommercial-Share Alike 2.5 Australia License.

% This is a chapter


\begin{bibunit}

% Argument and background information the user requires to read and understand the book
\chapter{The Comparative Measure of Problem Difficulty}
\label{chap:difficulty-measure}
% welcome
%\emph{Welcome to Clever Algorithms!} This is a handbook of recipes for computational problem solving techniques from the fields of Computational Intelligence, Biologically Inspired Computation, and Metaheuristics. 
% they are for using
%Clever Algorithms are interesting, practical, and fun to learn about and implement.
% briefly the audience
%Research scientists may be interested in browsing algorithm inspirations in search of an interesting system or process analogs to investigate. Developers and software engineers may compare various problem solving algorithms and technique-specific guidelines. Practitioners, students, and interested amateurs may implement state-of-the-art algorithms to address business or scientific needs, or simply play with the fascinating systems they represent.

% briefly book overview
%This introductory chapter provides relevant background information on Artificial Intelligence and Algorithms. The core of the book provides a large corpus of algorithms presented in a complete and consistent manner. The final chapter covers some advanced topics to consider once a number of algorithms have been mastered. This book has been designed as a reference text, where specific techniques are looked up, or where the algorithms across whole fields of study can be browsed, rather than being read cover-to-cover. This book is an algorithm handbook and a technique guidebook, and I hope you find something useful.


% 
% What is AI
% 
\section{What is AI}
\label{intro:sec:what_is_ai}
% 
% Artificial Intelligence (based on copy from my thesis)
% 
\subsection{Artificial Intelligence}
\label{sec:artificial_intelligence}
\index{Artificial Intelligence}


% change title of bib to references

\renewcommand{\bibsection}{\section{\bibname}}
\putbib
\end{bibunit}
