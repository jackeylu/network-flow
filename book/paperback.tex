% !TEX TS-program = xelatex
% !TEX encoding = UTF-8 
%
% (c) Copyright 2012 Lin Lu. Some Rights Reserved. 
% This work is licensed under a Creative Commons Attribution-Share Alike 2.5 China Mainland License.

% paperback.tex

\documentclass[twoside, openright]{book}
% The Clever Algorithms Project: http://www.CleverAlgorithms.com
% (c) Copyright 2010 Jason Brownlee. Some Rights Reserved. 
% This work is licensed under a Creative Commons Attribution-Noncommercial-Share Alike 2.5 Australia License.


% 
% Definitions
% 

% The main title of the book
\newcommand{\mybooktitle}{Network Algorithms and Combinatorial Optimization}
% The sub title of the book
\newcommand{\mybooksubtitle}{读书笔记}
% title
\newcommand{\mybookauthor}{吕林}
% date
\newcommand{\mybookdate}{2012}


% new macro for starting a new page and changing the style to empty
% \newpage == ends the current page. 
% \thispagestyle == works in the same manner as the \pagestyle, except that it changes the style for the current page only. 
% empty == Produces empty heads and feet - no page numbers
\newcommand{\blanknonumber}{\newpage\thispagestyle{empty}}


% 
% Packages
% 


% a replacement for fancyheadings
% http://www.ctan.org/tex-archive/macros/latex/contrib/fancyhdr/
\usepackage{fancyhdr}
\fancyhead[LO]{\slshape\nouppercase{\leftmark}}
\fancyhead[RE]{\slshape\nouppercase{\rightmark}}
\fancyhead[LE,RO]{\thepage}
\renewcommand{\headrulewidth}{0.4pt}
\renewcommand{\sectionmark}[1]{\markright{\thesection.\ #1}}
\renewcommand{\chaptermark}[1]{\markboth{第\thechapter 章 \ #1.}{}}


\usepackage{ctex}%加上这么一行,并用xelatex编译,即可支持中文


% add an index
% http://ctan.unsw.edu.au/macros/latex/contrib/index/index.pdf
\usepackage{index} 

% Flexible and easy interface to page dimensions
% http://www.ctan.org/tex-archive/macros/latex/contrib/geometry/
% also, bigger pages by default
\usepackage{geometry}

% Supports the Text Companion fonts which provide many text symbols (such as baht, bullet, copyright, musicalnote, onequarter, section, and yen) in the TS1 encoding.
% http://www.ctan.org/tex-archive/help/Catalogue/entries/textcomp.html
% needed for listings
\usepackage{textcomp}

% http://www.maths.adelaide.edu.au/anthony.roberts/LaTeX/ltxusecol.html
% needed for listings - lots of pretty colors
\usepackage[usenames,dvipsnames]{color}

% bold in ttfamily
\usepackage{bold-extra}

% better spacing
% http://ctan.unsw.edu.au/macros/latex/contrib/microtype/microtype.pdf
\usepackage{microtype}

% code listings (lots of languages)
% http://mirror.aarnet.edu.au/pub/CTAN/macros/latex/contrib/listings/
\usepackage{listings} 
% define the look of the ruby code
% \lstset{language=ruby, 
%   basicstyle=\footnotesize\ttfamily, 
%   numbers=left, 
%   numberstyle=\tiny, 
%   frame=single, 
%   columns=flexible, 
%   upquote=true, 
%   showstringspaces=false, 
%   tabsize=2, 
%   captionpos=b,
%   breaklines=true,
%   breakatwhitespace=true,
%   keywordstyle=\color{blue}, 
%   stringstyle=\color{ForestGreen}, 
%   commentstyle=\color{Gray}}

% http://mirror.aarnet.edu.au/pub/CTAN/macros/latex/contrib/listings/
\lstset{language=ruby, 
	basicstyle=\footnotesize\ttfamily, 
  numbers=left, 
  numberstyle=\tiny, 
	keywordstyle=\bfseries\ttfamily,
  frame=single, 
  columns=flexible, 
  upquote=true, 
  showstringspaces=false, 
  tabsize=2, 
  captionpos=b,
  breaklines=true,
  breakatwhitespace=true}

% for ebooks, turn cross references into links
% http://www.tug.org/applications/hyperref/manual.html
\usepackage[
	breaklinks=true,
	colorlinks=true,
	urlcolor=blue,
	linkcolor=blue,
	citecolor=blue]{hyperref}

% modifies the widths of certain columns, rather than the inter column space, to set a table with the requested total width
% http://www.cs.brown.edu/system/software/latex/doc/tabularx.pdf
\usepackage{tabularx}

% This package provide some additional commands to enhance the quality of tables in LaTeX.
% http://www.ctan.org/tex-archive/macros/latex/contrib/booktabs/
\usepackage{booktabs}

% for multirows
\usepackage{multirow}

% a form of verbatim command that allows linebreaks at certain characters or combinations of characters
% http://www.tex.ac.uk/tex-archive/help/Catalogue/entries/url.html
% works well with hyperref
\usepackage{url}

% Algorithm2e is an environment for writing algorithms in LaTeX2e
% http://www.ctan.org/tex-archive/macros/latex/contrib/algorithm2e/
\usepackage[algoruled, linesnumbered, algosection]{styles/algorithm2e}

% for adding in bib for each section or chapter
% http://merkel.zoneo.net/Latex/natbib.php
\usepackage[numbers, sort&compress]{natbib}
\usepackage{styles/bibunits}

% maths
\usepackage{amsmath}
\usepackage{latexsym}

% graphics
\usepackage{graphicx}


% The Clever Algorithms Project: http://www.CleverAlgorithms.com
% (c) Copyright 2010 Jason Brownlee. Some Rights Reserved. 
% This work is licensed under a Creative Commons Attribution-Noncommercial-Share Alike 2.5 Australia License.


% Title Page


% def
\makeatletter
\def\thickhrulefill{\leavevmode \leaders \hrule height 1pt\hfill \kern \z@}
\renewcommand{\maketitle}{\begin{titlepage}%
    \let\footnotesize\small
    \let\footnoterule\relax
    \parindent \z@
    \reset@font
    \null
    \vskip 10\p@
    \hbox{\mbox{\hspace{3em}}%
      \vrule depth 0.6\textheight%
      \mbox{\hspace{2em}}
      \vbox{
        \vskip 40\p@
        \begin{flushleft}
          \Large \@author \par
        \end{flushleft}
        \vskip 70\p@
        \begin{flushleft}
          \huge \bfseries \@title \par
        \end{flushleft}
        \vfil
        }}
    \null
  \end{titlepage}%
  \setcounter{footnote}{0}%
}
\makeatother
\author{\mybookauthor}
\title{\mybooktitle\\{\large\mybooksubtitle}}
\date{\mybookdate}
\makeindex

% paperback only
\geometry{bindingoffset=1cm, twoside, paperwidth=6in, paperheight=9in, top=.9in, bottom=.6667in, outer=0.667in}

%\includeonly{f_copyright}

\begin{document}
\defaultbibliography{../workspace/bibtex}
\frontmatter
	\maketitle
	% (c) Copyright 2012 Lin Lu. Some Rights Reserved. 
% This work is licensed under a Creative Commons Attribution-Share Alike 2.5 China Mainland License.

% copyright.tex

\vspace*{\fill}
\begin{flushleft}
\begin{small}

% The Cover
\subsubsection*{吕林, PhD}
%Jason Brownlee studied Applied Science at Swinburne University in Melbourne, Australia, going on to complete a Masters in Information Technology focusing on Niching Genetic Algorithms, and a PhD in the field of Artificial Immune Systems. Jason has worked for a number of years as a Consultant and Software Engineer for a range of Corporate and Government organizations. When not writing books, Jason likes to compete in Machine Learning competitions.

% The Cover
\subsubsection*{Cover Image}
\copyright\ Copyright \mybookdate\ \mybookauthor. All Reserved. \\
\vspace{0.5cm}

% The Book
\subsubsection*{\mybooktitle: \mybooksubtitle}
\copyright\ Copyright \mybookdate\ \mybookauthor. Some Rights Reserved. \\
\vspace{0.5cm}

First Edition. XXXXX. January 2012 \\
ISBN: XXX-X-XXXX-XXXX-X \\
\vspace{0.5cm}

This work is licensed under a Creative Commons Attribution \--Share Alike 2.5 China Mainland License. \\
The full terms of the license are located online at \url{http://creativecommons.org/licenses/by-sa/2.5/cn/deed.en} \\
\vspace{0.5cm}

\subsubsection*{Webpage}
Source code and additional resources can be downloaded from the books companion website online at \url{http://github.com/jackeylu/network-flow}

\end{small}
\end{flushleft}

		%版权申明
	\cleardoublepage% toc

\setcounter{tocdepth}{1}
\tableofcontents
	\addtocontents{toc}{\protect\markboth{Contents}{}}
	%\cleardoublepage% 
% Foreword
% 
\chapter*{Foreword\markboth{Foreword}{}}
\addcontentsline{toc}{chapter}{Foreword}

I am delighted to write this foreword. This book, a reference where one can look up the details of most any algorithm to find a clear unambiguous description, has long been needed and here it finally is.  A concise reference that has taken many hours to write but which has the capacity to save vast amounts of time previously spent digging out original papers.

I have known the author for several years and have had experience of his amazing capacity for work and the sheer quality of his output, so this book comes as no surprise to me.  But I hope it will be a surprise and delight to you, the reader for whom it has been written.

But useful as this book is, it is only a beginning. There are so many algorithms that no one author could hope to cover them all. So if you know of an algorithm that is not yet here, how about contributing it using the same clear and lucid style?

\begin{flushright}
\vspace{0.5in}
Professor Tim Hendtlass \\
Complex Intelligent Systems Laboratory  \\
Faculty of Information and Communication Technologies  \\
Swinburne University of Technology
\end{flushright}

\begin{flushleft}
\vspace{0.2in}
Melbourne, Australia \\
2010
\end{flushleft}
%序
	\cleardoublepage% The Clever Algorithms Project: http://www.CleverAlgorithms.com
% (c) Copyright 2010 Jason Brownlee. Some Rights Reserved. 
% This work is licensed under a Creative Commons Attribution-Noncommercial-Share Alike 2.5 Australia License.


% A preface generally covers the story of how the book came into being, or how the idea for the book was developed; this is often followed by thanks and acknowledgments to people who were helpful to the author during the time of writing.
\chapter*{前言}

% 
% About the book
% 
\section*{关于此书}

此书系本人针对徐慧颖老师提出的网络算法与组合优化问题求解系列算法、问题的学习
心得,主要用于本人日后回忆,如对诸君能有所帮助,将不胜荣幸。

\begin{flushright}
\vspace{1in}
吕林
\end{flushright}

\begin{flushleft}
\vspace{0.2in}
珞珈山, 武汉大学 \\
2012
\end{flushleft}
%前言
	\cleardoublepage% 
% Acknowledgments
% 
\chapter*{致谢\markboth{致谢}{}}
% general
This book could not have been completed without the commitment, passion, and hard work from a large group of editors and supporters.

% Steve
%A special thanks to Steve Dower for his incredible attention to detail in providing technical and copy edits for large portions of this book, and for his enthusiasm for the subject area.
% Dan
%Also, a special thanks to Daniel Angus for the discussions around the genesis of the project, his continued support with the idea of an `algorithms atlas' and for his attention to detail in providing technical and copy edits for key chapters.

% helpers and technical editors
%In no particular order, thanks to:
%Juan Ojeda, 
%Martin Goddard, 
%David Howden, 
%Sean Luke, 
%David Zappia, 
%Jeremy Wazny, 
%Andrew Murray,
%John Wise,
%Patrick Boehnke, 
%Martin-Louis Bright,
%Leif Wickland,
%Andrew Myers,
%Paul Chinnery,
%Donald Doherty,
%Brook Tamir,
%Zach Scott,
%Diego Noble,
%Jason Davies,
%Mark Chenoweth,
%Markus Stokmaier,
%and 
%Stefan Pauleweit.

% cover voting
%Thanks to the hundreds of machine learning enthusiasts who voted on potential covers and helped shape what this book became. You know who you are!

% support
%Finally, I would like to thank my beautiful wife Ying Liu for her unrelenting support and patience throughout the project.
%致谢	
\mainmatter 
	\part{图论相关算法基础知识}
		% (c) Copyright 2012 Lin Lu. Some Rights Reserved. 
% This work is licensed under a Creative Commons Attribution-Share Alike 2.5 China Mainland License.

% This is a chapter


% Argument and background information the user requires to read and understand the book
\chapter{图的算法基本概念}
\label{chap:graph-primer}

图(graph)是由对象(称为顶点)集合及它们之间的连接集合组成。图的应用领域很广,
包括绘图(地理信息系统)、运输(道路网络和航班网络)、电子工程(电路中)和计算网络
(因特网互联中)。


% 
% What is graph
% 
\section{图的表示}
\label{intro:sec:graph:presentation}

Graphs常见的可以分为undirected graphs(无向图), directed graphs(有向图),
Weighted graphs(权重图), Hypergraphs(混合图). 前三者都好理解,最后一种
则是指multiple relationships that may exist between the same two vertices
($u$, $v$)。

\subsection{图的抽象数据类型}
图G可看做由顶点(vertex)集合V和V中顶点对构成的边(edge)集合E组成。因此,图是表示
集合V中对象之间的连接和关系的一种方式。而在一些书籍中,利用图的不同术语,
把顶点称为节点(node),边称为弧(arc)。

图中的边分为有向(directed),和无向的(undirected)。如果$(u,v)$对是有序的,且$u$
是$v$的前驱,则称从$u$到$v$的边$(u,v)$是有向的。如果$(u,v)$对是无序的,则称
从$u$到$v$的边$(u,v)$是无向的。无向边有时用集合$\{u,v\}$表示,但简单起见,
用$(u,v)$表示。对于无向的情况,$(u,v)$与$(v,u)$相同。图中通常把顶点可视化为
圆或矩形,把边可视化为线段或曲线,用于把圆或矩形连接起来。

\textbf{连通图}: If a path exists between any two pairs of vertices in a 
graph, then that graph is \textit{connected}.
% 
% Artificial Intelligence (based on copy from my thesis)
% 
\subsection{标准数据结构}
\label{sec:standard_data_structures}
\index{Standard Data Structures}
图的常见数据结构有两种:
邻接链表(adjacency list)和邻接矩阵(adjacency matrix).

\subsection{Storage Issues}
Adjacency matrix using a two-dimensional matrix to represent potential 
relationships among $n$ elements in a set. First, the matrix requires
$n^2$ elements of storage, yet there are times when the number of 
relationships is much smaller(当是稀疏矩阵的时候,空间利用率很低).
In these cases \-- known as \textit{sparse} graphs(稀疏图) \--it may be
impossible to store large graphs with more than several thousand vertices
because of the limitations of computer memory. 
Second, matrices are unsuitable when there may be multiple relationships
between a pair of elements. To store there relationships in a matrix,
each element would become a list, and the abstraction of $A[i][j]$ being
the $ij^{th}$ element breaks.

关于稀疏图,有个例子。2005年,国际机场协会(Airports Council International,
ACI)报导说全球一共有1659个机场,用邻接矩阵存储,则是2,752,281个矩阵元素。
那么“How many of these entries has a value?”大概是这些机场利用率如何?
ACI报导说2005年共计有71.6 million 的航空活动,简单且粗暴地转换,则可以认为
每天有(71.6 million/365)= 196,164次航班,即使认为这些航班都是直达的,也就是
中途不会在别的机场停下来,这整个的机场利用率也只有(196,164/2,752,281 = 0.07127324571873293460951116546603),空闲率达到了93\%,因此,这是一个典型的
稀疏图实例。

\subsection{Graph Analysis}
在设计与图相关的算法的时候,必须要考虑合适的数据结构,常见的是邻接链表和邻接
矩阵,具体选用哪一种,关键在于图是否是一个稀疏图。

% TODO fix the length of table
\newcommand{\minitab}[2][1]{\begin{tabular}{#1}#2\end{tabular}}
\begin{table}
\caption{Performance comparison of two algorithm variations}
\label{tab:per:compar:graph:variations}
\begin{tabularx}{12cm}{X|l|X|l}
\toprule
Graph type & O((V+E)*logV) & Comparison & O($V^2$+E) \\
\midrule
\minitab[c] {Sparse graph: \\ E is O(V)}
 & O(V log V) & is smaller than & O($V^2$) \\
\minitab[c]{Break-even graph:\\ E is O($V^2$/log V)}
 & O($V^2$+V*log V)=O($V^2$) & is equivalent to & O($V^2 + V^2/log V$) = O($V^2$) \\
\minitab[c] {Dense graph: \\ E is O($V^2$) } & O($V^2 log V$) 
& is larger than & O($V^2$)\\
\bottomrule
\end{tabularx}
\end{table}
% change title of bib to references


	\part{Simple network problems and algorithms: The “pretty”}%第一部分
		% The Clever Algorithms Project: http://www.CleverAlgorithms.com
% (c) Copyright 2010 Jason Brownlee. Some Rights Reserved. 
% This work is licensed under a Creative Commons Attribution-Noncommercial-Share Alike 2.5 Australia License.

% This is a chapter


\begin{bibunit}

% Argument and background information the user requires to read and understand the book
\chapter{Network Connectivity}
\label{chap:network-connectivity}
% welcome
%\emph{Welcome to Clever Algorithms!} This is a handbook of recipes for computational problem solving techniques from the fields of Computational Intelligence, Biologically Inspired Computation, and Metaheuristics. 
% they are for using
%Clever Algorithms are interesting, practical, and fun to learn about and implement.
% briefly the audience
%Research scientists may be interested in browsing algorithm inspirations in search of an interesting system or process analogs to investigate. Developers and software engineers may compare various problem solving algorithms and technique-specific guidelines. Practitioners, students, and interested amateurs may implement state-of-the-art algorithms to address business or scientific needs, or simply play with the fascinating systems they represent.

% briefly book overview
%This introductory chapter provides relevant background information on Artificial Intelligence and Algorithms. The core of the book provides a large corpus of algorithms presented in a complete and consistent manner. The final chapter covers some advanced topics to consider once a number of algorithms have been mastered. This book has been designed as a reference text, where specific techniques are looked up, or where the algorithms across whole fields of study can be browsed, rather than being read cover-to-cover. This book is an algorithm handbook and a technique guidebook, and I hope you find something useful.


% 
% What is AI
% 
\section{What is AI}
\label{intro:sec:what_is_ai}
% 
% Artificial Intelligence (based on copy from my thesis)
% 
\subsection{Artificial Intelligence}
\label{sec:artificial_intelligence}
\index{Artificial Intelligence}


% change title of bib to references

\renewcommand{\bibsection}{\section{\bibname}}
\putbib
\end{bibunit}
		%引言
		% The Clever Algorithms Project: http://www.CleverAlgorithms.com
% (c) Copyright 2010 Jason Brownlee. Some Rights Reserved. 
% This work is licensed under a Creative Commons Attribution-Noncommercial-Share Alike 2.5 Australia License.

% This is a chapter


\chapter{Shortest Path}
\label{chap:Shortest-Path}
% welcome
%\emph{Welcome to Clever Algorithms!} This is a handbook of recipes for computational problem solving techniques from the fields of Computational Intelligence, Biologically Inspired Computation, and Metaheuristics. 
% they are for using
%Clever Algorithms are interesting, practical, and fun to learn about and implement.
% briefly the audience
%Research scientists may be interested in browsing algorithm inspirations in search of an interesting system or process analogs to investigate. Developers and software engineers may compare various problem solving algorithms and technique-specific guidelines. Practitioners, students, and interested amateurs may implement state-of-the-art algorithms to address business or scientific needs, or simply play with the fascinating systems they represent.

% briefly book overview
%This introductory chapter provides relevant background information on Artificial Intelligence and Algorithms. The core of the book provides a large corpus of algorithms presented in a complete and consistent manner. The final chapter covers some advanced topics to consider once a number of algorithms have been mastered. This book has been designed as a reference text, where specific techniques are looked up, or where the algorithms across whole fields of study can be browsed, rather than being read cover-to-cover. This book is an algorithm handbook and a technique guidebook, and I hope you find something useful.


% 
% What is AI
% 
\section{Shortest Path}
\label{intro:sec:what_is_ai}
% 
% Artificial Intelligence (based on copy from my thesis)
% 
\subsection{Artificial Intelligence}
\label{sec:artificial_intelligence}
\index{Artificial Intelligence}


% change title of bib to references


		% The Clever Algorithms Project: http://www.CleverAlgorithms.com
% (c) Copyright 2010 Jason Brownlee. Some Rights Reserved. 
% This work is licensed under a Creative Commons Attribution-Noncommercial-Share Alike 2.5 Australia License.

% This is a chapter


\begin{bibunit}

% Argument and background information the user requires to read and understand the book
\chapter{Max flow and min cut}
\label{chap:Max-flow-and-min-cut}
% welcome
%\emph{Welcome to Clever Algorithms!} This is a handbook of recipes for computational problem solving techniques from the fields of Computational Intelligence, Biologically Inspired Computation, and Metaheuristics. 
% they are for using
%Clever Algorithms are interesting, practical, and fun to learn about and implement.
% briefly the audience
%Research scientists may be interested in browsing algorithm inspirations in search of an interesting system or process analogs to investigate. Developers and software engineers may compare various problem solving algorithms and technique-specific guidelines. Practitioners, students, and interested amateurs may implement state-of-the-art algorithms to address business or scientific needs, or simply play with the fascinating systems they represent.

% briefly book overview
%This introductory chapter provides relevant background information on Artificial Intelligence and Algorithms. The core of the book provides a large corpus of algorithms presented in a complete and consistent manner. The final chapter covers some advanced topics to consider once a number of algorithms have been mastered. This book has been designed as a reference text, where specific techniques are looked up, or where the algorithms across whole fields of study can be browsed, rather than being read cover-to-cover. This book is an algorithm handbook and a technique guidebook, and I hope you find something useful.


% 
% What is AI
% 
\section{What is AI}
\label{intro:sec:what_is_ai}
% 
% Artificial Intelligence (based on copy from my thesis)
% 
\subsection{Artificial Intelligence}
\label{sec:artificial_intelligence}
\index{Artificial Intelligence}


% change title of bib to references

\renewcommand{\bibsection}{\section{\bibname}}
\putbib
\end{bibunit}

		% The Clever Algorithms Project: http://www.CleverAlgorithms.com
% (c) Copyright 2010 Jason Brownlee. Some Rights Reserved. 
% This work is licensed under a Creative Commons Attribution-Noncommercial-Share Alike 2.5 Australia License.

% This is a chapter


\begin{bibunit}

% Argument and background information the user requires to read and understand the book
\chapter{Minimum Spanning Tree}
\label{chap:minimum-spanning-tree}
% welcome
%\emph{Welcome to Clever Algorithms!} This is a handbook of recipes for computational problem solving techniques from the fields of Computational Intelligence, Biologically Inspired Computation, and Metaheuristics. 
% they are for using
%Clever Algorithms are interesting, practical, and fun to learn about and implement.
% briefly the audience
%Research scientists may be interested in browsing algorithm inspirations in search of an interesting system or process analogs to investigate. Developers and software engineers may compare various problem solving algorithms and technique-specific guidelines. Practitioners, students, and interested amateurs may implement state-of-the-art algorithms to address business or scientific needs, or simply play with the fascinating systems they represent.

% briefly book overview
%This introductory chapter provides relevant background information on Artificial Intelligence and Algorithms. The core of the book provides a large corpus of algorithms presented in a complete and consistent manner. The final chapter covers some advanced topics to consider once a number of algorithms have been mastered. This book has been designed as a reference text, where specific techniques are looked up, or where the algorithms across whole fields of study can be browsed, rather than being read cover-to-cover. This book is an algorithm handbook and a technique guidebook, and I hope you find something useful.


% 
% What is AI
% 
\section{What is AI}
\label{intro:sec:what_is_ai}
% 
% Artificial Intelligence (based on copy from my thesis)
% 
\subsection{Artificial Intelligence}
\label{sec:artificial_intelligence}
\index{Artificial Intelligence}


% change title of bib to references

\renewcommand{\bibsection}{\section{\bibname}}
\putbib
\end{bibunit}

		% The Clever Algorithms Project: http://www.CleverAlgorithms.com
% (c) Copyright 2010 Jason Brownlee. Some Rights Reserved. 
% This work is licensed under a Creative Commons Attribution-Noncommercial-Share Alike 2.5 Australia License.

% This is a chapter


\begin{bibunit}

% Argument and background information the user requires to read and understand the book
\chapter{Maximum Matching}
\label{chap:maximum-matching}
% welcome
%\emph{Welcome to Clever Algorithms!} This is a handbook of recipes for computational problem solving techniques from the fields of Computational Intelligence, Biologically Inspired Computation, and Metaheuristics. 
% they are for using
%Clever Algorithms are interesting, practical, and fun to learn about and implement.
% briefly the audience
%Research scientists may be interested in browsing algorithm inspirations in search of an interesting system or process analogs to investigate. Developers and software engineers may compare various problem solving algorithms and technique-specific guidelines. Practitioners, students, and interested amateurs may implement state-of-the-art algorithms to address business or scientific needs, or simply play with the fascinating systems they represent.

% briefly book overview
%This introductory chapter provides relevant background information on Artificial Intelligence and Algorithms. The core of the book provides a large corpus of algorithms presented in a complete and consistent manner. The final chapter covers some advanced topics to consider once a number of algorithms have been mastered. This book has been designed as a reference text, where specific techniques are looked up, or where the algorithms across whole fields of study can be browsed, rather than being read cover-to-cover. This book is an algorithm handbook and a technique guidebook, and I hope you find something useful.


% 
% What is AI
% 
\section{What is AI}
\label{intro:sec:what_is_ai}
% 
% Artificial Intelligence (based on copy from my thesis)
% 
\subsection{Artificial Intelligence}
\label{sec:artificial_intelligence}
\index{Artificial Intelligence}


% change title of bib to references

\renewcommand{\bibsection}{\section{\bibname}}
\putbib
\end{bibunit}

			
%Simple network problems and algorithms: The “pretty”
%Difficult network problems
%Computational complexity: The “theory”
%Linear and integer programming: The “machinery”
%Heuristic algorithms: The “ugly”

	\part{Difficult network problems} %第二部分 算法
% page 24
%Looks simple but no polynomial algorithms exist (yet?)
%The traveling salesman problem
%Maximal independent set
%Capacited minimal spanning tree
%Multi-commodity network (integral) flow

%		\include{c_stochastic}
%		\include{c_evolution}
%		\include{c_physical}
%		\include{c_probabilistic}
%		\include{c_swarm}
%		\include{c_immune}
%		\include{c_neural}
	\part{Computational complexity: The “theory”}
		% The Clever Algorithms Project: http://www.CleverAlgorithms.com
% (c) Copyright 2010 Jason Brownlee. Some Rights Reserved. 
% This work is licensed under a Creative Commons Attribution-Noncommercial-Share Alike 2.5 Australia License.

% This is a chapter


\begin{bibunit}

% Argument and background information the user requires to read and understand the book
\chapter{The Theory of NP-Completeness}
\label{chap:NPC}
% welcome
%\emph{Welcome to Clever Algorithms!} This is a handbook of recipes for computational problem solving techniques from the fields of Computational Intelligence, Biologically Inspired Computation, and Metaheuristics. 
% they are for using
%Clever Algorithms are interesting, practical, and fun to learn about and implement.
% briefly the audience
%Research scientists may be interested in browsing algorithm inspirations in search of an interesting system or process analogs to investigate. Developers and software engineers may compare various problem solving algorithms and technique-specific guidelines. Practitioners, students, and interested amateurs may implement state-of-the-art algorithms to address business or scientific needs, or simply play with the fascinating systems they represent.

% briefly book overview
%This introductory chapter provides relevant background information on Artificial Intelligence and Algorithms. The core of the book provides a large corpus of algorithms presented in a complete and consistent manner. The final chapter covers some advanced topics to consider once a number of algorithms have been mastered. This book has been designed as a reference text, where specific techniques are looked up, or where the algorithms across whole fields of study can be browsed, rather than being read cover-to-cover. This book is an algorithm handbook and a technique guidebook, and I hope you find something useful.


% 
% What is AI
% 
\section{What is AI}
\label{intro:sec:what_is_ai}
% 
% Artificial Intelligence (based on copy from my thesis)
% 
\subsection{Artificial Intelligence}
\label{sec:artificial_intelligence}
\index{Artificial Intelligence}


% change title of bib to references

\renewcommand{\bibsection}{\section{\bibname}}
\putbib
\end{bibunit}

		% The Clever Algorithms Project: http://www.CleverAlgorithms.com
% (c) Copyright 2010 Jason Brownlee. Some Rights Reserved. 
% This work is licensed under a Creative Commons Attribution-Noncommercial-Share Alike 2.5 Australia License.

% This is a chapter


\begin{bibunit}

% Argument and background information the user requires to read and understand the book
\chapter{The Comparative Measure of Problem Difficulty}
\label{chap:difficulty-measure}
% welcome
%\emph{Welcome to Clever Algorithms!} This is a handbook of recipes for computational problem solving techniques from the fields of Computational Intelligence, Biologically Inspired Computation, and Metaheuristics. 
% they are for using
%Clever Algorithms are interesting, practical, and fun to learn about and implement.
% briefly the audience
%Research scientists may be interested in browsing algorithm inspirations in search of an interesting system or process analogs to investigate. Developers and software engineers may compare various problem solving algorithms and technique-specific guidelines. Practitioners, students, and interested amateurs may implement state-of-the-art algorithms to address business or scientific needs, or simply play with the fascinating systems they represent.

% briefly book overview
%This introductory chapter provides relevant background information on Artificial Intelligence and Algorithms. The core of the book provides a large corpus of algorithms presented in a complete and consistent manner. The final chapter covers some advanced topics to consider once a number of algorithms have been mastered. This book has been designed as a reference text, where specific techniques are looked up, or where the algorithms across whole fields of study can be browsed, rather than being read cover-to-cover. This book is an algorithm handbook and a technique guidebook, and I hope you find something useful.


% 
% What is AI
% 
\section{What is AI}
\label{intro:sec:what_is_ai}
% 
% Artificial Intelligence (based on copy from my thesis)
% 
\subsection{Artificial Intelligence}
\label{sec:artificial_intelligence}
\index{Artificial Intelligence}


% change title of bib to references

\renewcommand{\bibsection}{\section{\bibname}}
\putbib
\end{bibunit}

		%% The Clever Algorithms Project: http://www.CleverAlgorithms.com
% (c) Copyright 2010 Jason Brownlee. Some Rights Reserved. 
% This work is licensed under a Creative Commons Attribution-Noncommercial-Share Alike 2.5 Australia License.

% This is a chapter


\begin{bibunit}

% Argument and background information the user requires to read and understand the book
\chapter{The Theory of NP-Completeness}
\label{chap:NPC}
% welcome
%\emph{Welcome to Clever Algorithms!} This is a handbook of recipes for computational problem solving techniques from the fields of Computational Intelligence, Biologically Inspired Computation, and Metaheuristics. 
% they are for using
%Clever Algorithms are interesting, practical, and fun to learn about and implement.
% briefly the audience
%Research scientists may be interested in browsing algorithm inspirations in search of an interesting system or process analogs to investigate. Developers and software engineers may compare various problem solving algorithms and technique-specific guidelines. Practitioners, students, and interested amateurs may implement state-of-the-art algorithms to address business or scientific needs, or simply play with the fascinating systems they represent.

% briefly book overview
%This introductory chapter provides relevant background information on Artificial Intelligence and Algorithms. The core of the book provides a large corpus of algorithms presented in a complete and consistent manner. The final chapter covers some advanced topics to consider once a number of algorithms have been mastered. This book has been designed as a reference text, where specific techniques are looked up, or where the algorithms across whole fields of study can be browsed, rather than being read cover-to-cover. This book is an algorithm handbook and a technique guidebook, and I hope you find something useful.


% 
% What is AI
% 
\section{What is AI}
\label{intro:sec:what_is_ai}
% 
% Artificial Intelligence (based on copy from my thesis)
% 
\subsection{Artificial Intelligence}
\label{sec:artificial_intelligence}
\index{Artificial Intelligence}


% change title of bib to references

\renewcommand{\bibsection}{\section{\bibname}}
\putbib
\end{bibunit}

	\part{Linear and integer programming: The “machinery”}
		% The Clever Algorithms Project: http://www.CleverAlgorithms.com
% (c) Copyright 2010 Jason Brownlee. Some Rights Reserved. 
% This work is licensed under a Creative Commons Attribution-Noncommercial-Share Alike 2.5 Australia License.

% This is a chapter


\begin{bibunit}

% Argument and background information the user requires to read and understand the book
\chapter{Linear Programming}
\label{chap:linear-programming}
% welcome
%\emph{Welcome to Clever Algorithms!} This is a handbook of recipes for computational problem solving techniques from the fields of Computational Intelligence, Biologically Inspired Computation, and Metaheuristics. 
% they are for using
%Clever Algorithms are interesting, practical, and fun to learn about and implement.
% briefly the audience
%Research scientists may be interested in browsing algorithm inspirations in search of an interesting system or process analogs to investigate. Developers and software engineers may compare various problem solving algorithms and technique-specific guidelines. Practitioners, students, and interested amateurs may implement state-of-the-art algorithms to address business or scientific needs, or simply play with the fascinating systems they represent.

% briefly book overview
%This introductory chapter provides relevant background information on Artificial Intelligence and Algorithms. The core of the book provides a large corpus of algorithms presented in a complete and consistent manner. The final chapter covers some advanced topics to consider once a number of algorithms have been mastered. This book has been designed as a reference text, where specific techniques are looked up, or where the algorithms across whole fields of study can be browsed, rather than being read cover-to-cover. This book is an algorithm handbook and a technique guidebook, and I hope you find something useful.


% 
% What is AI
% 
\section{What is AI}
\label{intro:sec:what_is_ai}
% 
% Artificial Intelligence (based on copy from my thesis)
% 
\subsection{Artificial Intelligence}
\label{sec:artificial_intelligence}
\index{Artificial Intelligence}


% change title of bib to references

\renewcommand{\bibsection}{\section{\bibname}}
\putbib
\end{bibunit}

		% The Clever Algorithms Project: http://www.CleverAlgorithms.com
% (c) Copyright 2010 Jason Brownlee. Some Rights Reserved. 
% This work is licensed under a Creative Commons Attribution-Noncommercial-Share Alike 2.5 Australia License.

% This is a chapter


\begin{bibunit}

% Argument and background information the user requires to read and understand the book
\chapter{Integer Programming}
\label{chap:integer-programming}
% welcome
%\emph{Welcome to Clever Algorithms!} This is a handbook of recipes for computational problem solving techniques from the fields of Computational Intelligence, Biologically Inspired Computation, and Metaheuristics. 
% they are for using
%Clever Algorithms are interesting, practical, and fun to learn about and implement.
% briefly the audience
%Research scientists may be interested in browsing algorithm inspirations in search of an interesting system or process analogs to investigate. Developers and software engineers may compare various problem solving algorithms and technique-specific guidelines. Practitioners, students, and interested amateurs may implement state-of-the-art algorithms to address business or scientific needs, or simply play with the fascinating systems they represent.

% briefly book overview
%This introductory chapter provides relevant background information on Artificial Intelligence and Algorithms. The core of the book provides a large corpus of algorithms presented in a complete and consistent manner. The final chapter covers some advanced topics to consider once a number of algorithms have been mastered. This book has been designed as a reference text, where specific techniques are looked up, or where the algorithms across whole fields of study can be browsed, rather than being read cover-to-cover. This book is an algorithm handbook and a technique guidebook, and I hope you find something useful.


% 
% What is AI
% 
\section{What is AI}
\label{intro:sec:what_is_ai}
% 
% Artificial Intelligence (based on copy from my thesis)
% 
\subsection{Artificial Intelligence}
\label{sec:artificial_intelligence}
\index{Artificial Intelligence}


% change title of bib to references

\renewcommand{\bibsection}{\section{\bibname}}
\putbib
\end{bibunit}

	\part{Heuristic algorithms: The “ugly”}
		% The Clever Algorithms Project: http://www.CleverAlgorithms.com
% (c) Copyright 2010 Jason Brownlee. Some Rights Reserved. 
% This work is licensed under a Creative Commons Attribution-Noncommercial-Share Alike 2.5 Australia License.

% This is a chapter


\begin{bibunit}

% Argument and background information the user requires to read and understand the book
\chapter{Heuristic Algorithms}
\label{chap:heuristic-algorithms}
% welcome
%\emph{Welcome to Clever Algorithms!} This is a handbook of recipes for computational problem solving techniques from the fields of Computational Intelligence, Biologically Inspired Computation, and Metaheuristics. 
% they are for using
%Clever Algorithms are interesting, practical, and fun to learn about and implement.
% briefly the audience
%Research scientists may be interested in browsing algorithm inspirations in search of an interesting system or process analogs to investigate. Developers and software engineers may compare various problem solving algorithms and technique-specific guidelines. Practitioners, students, and interested amateurs may implement state-of-the-art algorithms to address business or scientific needs, or simply play with the fascinating systems they represent.

% briefly book overview
%This introductory chapter provides relevant background information on Artificial Intelligence and Algorithms. The core of the book provides a large corpus of algorithms presented in a complete and consistent manner. The final chapter covers some advanced topics to consider once a number of algorithms have been mastered. This book has been designed as a reference text, where specific techniques are looked up, or where the algorithms across whole fields of study can be browsed, rather than being read cover-to-cover. This book is an algorithm handbook and a technique guidebook, and I hope you find something useful.


% 
% What is AI
% 
\section{What is AI}
\label{intro:sec:what_is_ai}

\section{Local search algorithms}

\section{Definition of neighborhood}
\section{Techniques for avoiding local optima traps}
\index{Artificial Intelligence}



% change title of bib to references

\renewcommand{\bibsection}{\section{\bibname}}
\putbib
\end{bibunit}

	\part{附录}
		\appendix
		% This is an appendix

\chapter{Benchmark Dataset}
\label{ch:appendix1}
\index{Ruby}

\section{概要}
% this report


% basics
\section{基准数据集}
\index{Ruby!Language Basics}
This section summarizes the basics of the language, including variables, flow control, data structures, and functions.

\subsection{测试方法}
Ruby is an interpreted language, meaning that programs are typed as text into a \texttt{.rb} file which is parsed and executed at the time the script is run. For example, the following snippet shows how to invoke the Ruby interpreter on a script in the file \texttt{genetic\_algorithm.rb} from the command line: \texttt{ruby genetic\_algorithm.rb}

\newpage

这是

  % errata
	\cleardoublepage% 
% Errata
% 

\chapter*{勘误\markboth{勘误}{}}
\addcontentsline{toc}{part}{勘误}

\section*{First Edition, Revision 1}

\begin{small}
	
\begin{description}
	\item[page 9] Typo in Metaheuristics section of the Introduction. Thanks to Leif Wickland.
	
\end{description}

\end{small}
	
	
	% correct handling of the index (new odd page)
  \cleardoublepage
  \phantomsection
	\addcontentsline{toc}{part}{索引}
	{\footnotesize \printindex}	
	
	% blank page
	\blanknonumber\cleardoublepage
	
\end{document}
