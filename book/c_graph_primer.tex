% (c) Copyright 2012 Lin Lu. Some Rights Reserved. 
% This work is licensed under a Creative Commons Attribution-Share Alike 2.5 China Mainland License.

% This is a chapter


% Argument and background information the user requires to read and understand the book
\chapter{图的算法基本概念}
\label{chap:graph-primer}

图(graph)是由对象(称为顶点)集合及它们之间的连接集合组成。图的应用领域很广,
包括绘图(地理信息系统)、运输(道路网络和航班网络)、电子工程(电路中)和计算网络
(因特网互联中)。


% 
% What is graph
% 
\section{图的表示}
\label{intro:sec:graph:presentation}

Graphs常见的可以分为undirected graphs(无向图), directed graphs(有向图),
Weighted graphs(权重图), Hypergraphs(混合图). 前三者都好理解,最后一种
则是指multiple relationships that may exist between the same two vertices
($u$, $v$)。

\subsection{图的抽象数据类型}
图G可看做由顶点(vertex)集合V和V中顶点对构成的边(edge)集合E组成。因此,图是表示
集合V中对象之间的连接和关系的一种方式。而在一些书籍中,利用图的不同术语,
把顶点称为节点(node),边称为弧(arc)。

图中的边分为有向(directed),和无向的(undirected)。如果$(u,v)$对是有序的,且$u$
是$v$的前驱,则称从$u$到$v$的边$(u,v)$是有向的。如果$(u,v)$对是无序的,则称
从$u$到$v$的边$(u,v)$是无向的。无向边有时用集合$\{u,v\}$表示,但简单起见,
用$(u,v)$表示。对于无向的情况,$(u,v)$与$(v,u)$相同。图中通常把顶点可视化为
圆或矩形,把边可视化为线段或曲线,用于把圆或矩形连接起来。

\textbf{连通图}: If a path exists between any two pairs of vertices in a 
graph, then that graph is \textit{connected}.
% 
% Artificial Intelligence (based on copy from my thesis)
% 
\subsection{标准数据结构}
\label{sec:standard_data_structures}
\index{Standard Data Structures}
图的常见数据结构有两种:
邻接链表(adjacency list)和邻接矩阵(adjacency matrix).

\subsection{Storage Issues}
Adjacency matrix using a two-dimensional matrix to represent potential 
relationships among $n$ elements in a set. First, the matrix requires
$n^2$ elements of storage, yet there are times when the number of 
relationships is much smaller(当是稀疏矩阵的时候,空间利用率很低).
In these cases \-- known as \textit{sparse} graphs(稀疏图) \--it may be
impossible to store large graphs with more than several thousand vertices
because of the limitations of computer memory. 
Second, matrices are unsuitable when there may be multiple relationships
between a pair of elements. To store there relationships in a matrix,
each element would become a list, and the abstraction of $A[i][j]$ being
the $ij^{th}$ element breaks.

关于稀疏图,有个例子。2005年,国际机场协会(Airports Council International,
ACI)报导说全球一共有1659个机场,用邻接矩阵存储,则是2,752,281个矩阵元素。
那么“How many of these entries has a value?”大概是这些机场利用率如何?
ACI报导说2005年共计有71.6 million 的航空活动,简单且粗暴地转换,则可以认为
每天有(71.6 million/365)= 196,164次航班,即使认为这些航班都是直达的,也就是
中途不会在别的机场停下来,这整个的机场利用率也只有(196,164/2,752,281 = 0.07127324571873293460951116546603),空闲率达到了93\%,因此,这是一个典型的
稀疏图实例。

% change title of bib to references

