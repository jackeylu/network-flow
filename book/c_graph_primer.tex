% The Clever Algorithms Project: http://www.CleverAlgorithms.com
% (c) Copyright 2010 Jason Brownlee. Some Rights Reserved. 
% This work is licensed under a Creative Commons Attribution-Noncommercial-Share Alike 2.5 Australia License.

% This is a chapter


% Argument and background information the user requires to read and understand the book
\chapter{图的算法基本概念}
\label{chap:graph-primer}

图(graph)是由对象(称为顶点)集合及它们之间的连接集合组成。图的应用领域很广,
包括绘图(地理信息系统)、运输(道路网络和航班网络)、电子工程(电路中)和计算网络
(因特网互联中)。


% 
% What is AI
% 
\section{图的表示}
\label{intro:sec:graph:presentation}

\subsection{图的抽象数据类型}
图G可看做由顶点(vertex)集合V和V中顶点对构成的边(edge)集合E组成。因此,图是表示
集合V中对象之间的连接和关系的一种方式。而在一些书籍中,利用图的不同术语,
把顶点称为节点(node),边称为弧(arc)。

图中的边分为有向(directed),和无向的(undirected)。如果$(u,v)$对是有序的,且$u$
是$v$的前驱,则称从$u$到$v$的边$(u,v)$是有向的。如果$(u,v)$对是无序的,则称
从$u$到$v$的边$(u,v)$是无向的。无向边有时用集合$\{u,v\}$表示,但简单起见,
用$(u,v)$表示。对于无向的情况,$(u,v)$与$(v,u)$相同。图中通常把顶点可视化为
圆或矩形,把边可视化为线段或曲线,用于把圆或矩形连接起来。

% 
% Artificial Intelligence (based on copy from my thesis)
% 
\subsection{Artificial Intelligence}
\label{sec:artificial_intelligence}
\index{Artificial Intelligence}


% change title of bib to references

